%% Generated by Sphinx.
\def\sphinxdocclass{report}
\documentclass[letterpaper,10pt,english]{sphinxmanual}
\ifdefined\pdfpxdimen
   \let\sphinxpxdimen\pdfpxdimen\else\newdimen\sphinxpxdimen
\fi \sphinxpxdimen=.75bp\relax
\ifdefined\pdfimageresolution
    \pdfimageresolution= \numexpr \dimexpr1in\relax/\sphinxpxdimen\relax
\fi
%% let collapsible pdf bookmarks panel have high depth per default
\PassOptionsToPackage{bookmarksdepth=5}{hyperref}

\PassOptionsToPackage{booktabs}{sphinx}
\PassOptionsToPackage{colorrows}{sphinx}

\PassOptionsToPackage{warn}{textcomp}
\usepackage[utf8]{inputenc}
\ifdefined\DeclareUnicodeCharacter
% support both utf8 and utf8x syntaxes
  \ifdefined\DeclareUnicodeCharacterAsOptional
    \def\sphinxDUC#1{\DeclareUnicodeCharacter{"#1}}
  \else
    \let\sphinxDUC\DeclareUnicodeCharacter
  \fi
  \sphinxDUC{00A0}{\nobreakspace}
  \sphinxDUC{2500}{\sphinxunichar{2500}}
  \sphinxDUC{2502}{\sphinxunichar{2502}}
  \sphinxDUC{2514}{\sphinxunichar{2514}}
  \sphinxDUC{251C}{\sphinxunichar{251C}}
  \sphinxDUC{2572}{\textbackslash}
\fi
\usepackage{cmap}
\usepackage[T1]{fontenc}
\usepackage{amsmath,amssymb,amstext}
\usepackage{babel}



\usepackage{tgtermes}
\usepackage{tgheros}
\renewcommand{\ttdefault}{txtt}



\usepackage[Bjarne]{fncychap}
\usepackage{sphinx}

\fvset{fontsize=auto}
\usepackage{geometry}


% Include hyperref last.
\usepackage{hyperref}
% Fix anchor placement for figures with captions.
\usepackage{hypcap}% it must be loaded after hyperref.
% Set up styles of URL: it should be placed after hyperref.
\urlstyle{same}

\addto\captionsenglish{\renewcommand{\contentsname}{Contents:}}

\usepackage{sphinxmessages}
\setcounter{tocdepth}{1}



\title{Ping and Traceroute}
\date{Feb 28, 2025}
\release{}
\author{Cedric Bone}
\newcommand{\sphinxlogo}{\vbox{}}
\renewcommand{\releasename}{}
\makeindex
\begin{document}

\ifdefined\shorthandoff
  \ifnum\catcode`\=\string=\active\shorthandoff{=}\fi
  \ifnum\catcode`\"=\active\shorthandoff{"}\fi
\fi

\pagestyle{empty}
\sphinxmaketitle
\pagestyle{plain}
\sphinxtableofcontents
\pagestyle{normal}
\phantomsection\label{\detokenize{index::doc}}


\sphinxAtStartPar
Add your content using \sphinxcode{\sphinxupquote{reStructuredText}} syntax. See the
\sphinxhref{https://www.sphinx-doc.org/en/master/usage/restructuredtext/index.html}{reStructuredText}
documentation for details.

\sphinxstepscope


\chapter{my\_ping module}
\label{\detokenize{my_ping:module-my_ping}}\label{\detokenize{my_ping:my-ping-module}}\label{\detokenize{my_ping::doc}}\index{module@\spxentry{module}!my\_ping@\spxentry{my\_ping}}\index{my\_ping@\spxentry{my\_ping}!module@\spxentry{module}}
\sphinxAtStartPar
my\_ping.py

\sphinxAtStartPar
Python implementation of ping command.
Sends ICMP echo requests and measures round\sphinxhyphen{}trip time.
\begin{description}
\sphinxlineitem{Usage:}
\sphinxAtStartPar
python my\_ping.py {[}\sphinxhyphen{}c COUNT{]} {[}\sphinxhyphen{}i WAIT{]} {[}\sphinxhyphen{}s PACKETSIZE{]} {[}\sphinxhyphen{}t TIMEOUT{]} destination

\end{description}
\index{handle\_interrupt() (in module my\_ping)@\spxentry{handle\_interrupt()}\spxextra{in module my\_ping}}

\begin{fulllineitems}
\phantomsection\label{\detokenize{my_ping:my_ping.handle_interrupt}}
\pysigstartsignatures
\pysiglinewithargsret
{\sphinxcode{\sphinxupquote{my\_ping.}}\sphinxbfcode{\sphinxupquote{handle\_interrupt}}}
{\sphinxparam{\DUrole{n}{signum}}\sphinxparamcomma \sphinxparam{\DUrole{n}{frame}}\sphinxparamcomma \sphinxparam{\DUrole{n}{sent\_count}}\sphinxparamcomma \sphinxparam{\DUrole{n}{received\_count}}\sphinxparamcomma \sphinxparam{\DUrole{n}{rtts}}\sphinxparamcomma \sphinxparam{\DUrole{n}{destination}}}
{}
\pysigstopsignatures
\sphinxAtStartPar
Handle Ctrl+C
\begin{description}
\sphinxlineitem{Parameters:}
\sphinxAtStartPar
signum: Signal number
frame: Current frame
sent\_count: Total packets sent
received\_count: Total responses received
rtts: List of round\sphinxhyphen{}trip times
destination: Target IP

\end{description}

\end{fulllineitems}

\index{main() (in module my\_ping)@\spxentry{main()}\spxextra{in module my\_ping}}

\begin{fulllineitems}
\phantomsection\label{\detokenize{my_ping:my_ping.main}}
\pysigstartsignatures
\pysiglinewithargsret
{\sphinxcode{\sphinxupquote{my\_ping.}}\sphinxbfcode{\sphinxupquote{main}}}
{}
{}
\pysigstopsignatures
\sphinxAtStartPar
Parse command line arguments and do ping.
ICMP echo requests and displays results.

\end{fulllineitems}

\index{print\_stats() (in module my\_ping)@\spxentry{print\_stats()}\spxextra{in module my\_ping}}

\begin{fulllineitems}
\phantomsection\label{\detokenize{my_ping:my_ping.print_stats}}
\pysigstartsignatures
\pysiglinewithargsret
{\sphinxcode{\sphinxupquote{my\_ping.}}\sphinxbfcode{\sphinxupquote{print\_stats}}}
{\sphinxparam{\DUrole{n}{sent\_count}}\sphinxparamcomma \sphinxparam{\DUrole{n}{received\_count}}\sphinxparamcomma \sphinxparam{\DUrole{n}{rtts}}\sphinxparamcomma \sphinxparam{\DUrole{n}{destination}}}
{}
\pysigstopsignatures
\sphinxAtStartPar
Print ping summary
\begin{description}
\sphinxlineitem{Parameters:}
\sphinxAtStartPar
sent\_count: Total number sent
received\_count: Total number received
rtts: List of round\sphinxhyphen{}trip times
destination: Target hostname or IP

\end{description}

\end{fulllineitems}

\index{send\_ping() (in module my\_ping)@\spxentry{send\_ping()}\spxextra{in module my\_ping}}

\begin{fulllineitems}
\phantomsection\label{\detokenize{my_ping:my_ping.send_ping}}
\pysigstartsignatures
\pysiglinewithargsret
{\sphinxcode{\sphinxupquote{my\_ping.}}\sphinxbfcode{\sphinxupquote{send\_ping}}}
{\sphinxparam{\DUrole{n}{dest\_ip}}\sphinxparamcomma \sphinxparam{\DUrole{n}{timeout}}\sphinxparamcomma \sphinxparam{\DUrole{n}{packet\_size}}\sphinxparamcomma \sphinxparam{\DUrole{n}{sent\_count}}\sphinxparamcomma \sphinxparam{\DUrole{n}{received\_count}}}
{}
\pysigstopsignatures
\sphinxAtStartPar
Send a ping and wait for response
\begin{description}
\sphinxlineitem{Parameters:}
\sphinxAtStartPar
dest\_ip: Destination IP address
timeout: Maximum wait time for response
packet\_size: Size
sent\_count: Number of packets sent
received\_count: Number of responses received

\sphinxlineitem{Returns:}
\sphinxAtStartPar
tuple: (success\_bool, rtt\_ms)

\end{description}

\end{fulllineitems}


\sphinxstepscope


\chapter{my\_traceroute module}
\label{\detokenize{my_traceroute:module-my_traceroute}}\label{\detokenize{my_traceroute:my-traceroute-module}}\label{\detokenize{my_traceroute::doc}}\index{module@\spxentry{module}!my\_traceroute@\spxentry{my\_traceroute}}\index{my\_traceroute@\spxentry{my\_traceroute}!module@\spxentry{module}}
\sphinxAtStartPar
my\_traceroute.py

\sphinxAtStartPar
Python implementation of traceroute command.
Sends UDP probes with incrementing TTL values.
\begin{description}
\sphinxlineitem{Usage:}
\sphinxAtStartPar
sudo python my\_traceroute.py {[}\sphinxhyphen{}n{]} {[}\sphinxhyphen{}q NQUERIES{]} {[}\sphinxhyphen{}S{]} destination

\end{description}
\index{main() (in module my\_traceroute)@\spxentry{main()}\spxextra{in module my\_traceroute}}

\begin{fulllineitems}
\phantomsection\label{\detokenize{my_traceroute:my_traceroute.main}}
\pysigstartsignatures
\pysiglinewithargsret
{\sphinxcode{\sphinxupquote{my\_traceroute.}}\sphinxbfcode{\sphinxupquote{main}}}
{}
{}
\pysigstopsignatures
\sphinxAtStartPar
Parse arguments and execute traceroute
Shows network path with response times for each hop

\end{fulllineitems}

\index{send\_probe() (in module my\_traceroute)@\spxentry{send\_probe()}\spxextra{in module my\_traceroute}}

\begin{fulllineitems}
\phantomsection\label{\detokenize{my_traceroute:my_traceroute.send_probe}}
\pysigstartsignatures
\pysiglinewithargsret
{\sphinxcode{\sphinxupquote{my\_traceroute.}}\sphinxbfcode{\sphinxupquote{send\_probe}}}
{\sphinxparam{\DUrole{n}{send\_socket}}\sphinxparamcomma \sphinxparam{\DUrole{n}{recv\_socket}}\sphinxparamcomma \sphinxparam{\DUrole{n}{dest\_ip}}\sphinxparamcomma \sphinxparam{\DUrole{n}{ttl}}\sphinxparamcomma \sphinxparam{\DUrole{n}{port}}\sphinxparamcomma \sphinxparam{\DUrole{n}{timeout}}}
{}
\pysigstopsignatures
\sphinxAtStartPar
Send a probe and wait for response
\begin{description}
\sphinxlineitem{Parameters:}
\sphinxAtStartPar
send\_socket: Socket for probes
recv\_socket: Socket for responses
dest\_ip: Destination IP
ttl: Time\sphinxhyphen{}to\sphinxhyphen{}live
port: destination port
timeout: Maximum wait time for response

\sphinxlineitem{Returns:}
\sphinxAtStartPar
tuple: (responding\_ip, elapsed\_time\_ms)

\end{description}

\end{fulllineitems}



\renewcommand{\indexname}{Python Module Index}
\begin{sphinxtheindex}
\let\bigletter\sphinxstyleindexlettergroup
\bigletter{m}
\item\relax\sphinxstyleindexentry{my\_ping}\sphinxstyleindexpageref{my_ping:\detokenize{module-my_ping}}
\item\relax\sphinxstyleindexentry{my\_traceroute}\sphinxstyleindexpageref{my_traceroute:\detokenize{module-my_traceroute}}
\end{sphinxtheindex}

\renewcommand{\indexname}{Index}
\printindex
\end{document}